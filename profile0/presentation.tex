\subsection{Rationale}\label{rationale}

GitHub is used by an enormous variety of hackers, including us, so we
wanted to learn more about its origins

\section{Organizational Details}\label{organizational-details}

\begin{center}\rule{3in}{0.4pt}\end{center}

\begin{itemize}
\itemsep1pt\parskip0pt\parsep0pt
\item
  GitHub is a Corporation Registered in the State of Delaware
\item
  GitHub was founded, April 10th, 2008
\item
  GitHub's Founding Fathers:

  \begin{itemize}
  \itemsep1pt\parskip0pt\parsep0pt
  \item
    Tom Preston-Werner
  \item
    Chris Wanstrath
  \item
    PJ Hyett
  \end{itemize}
\end{itemize}

\begin{center}\rule{3in}{0.4pt}\end{center}

\begin{itemize}
\itemsep1pt\parskip0pt\parsep0pt
\item
  All of the original founders are still active with the exception of
  Tom Preston-Werner.

  \begin{itemize}
  \itemsep1pt\parskip0pt\parsep0pt
  \item
    He left GitHub in 2014 after confirmed
    \href{http://bits.blogs.nytimes.com/2014/04/21/github-founder-resigns-after-investigation/}{harassment
    allegations}
  \end{itemize}
\item
  GitHub is a private corporation and isn't publically traded
\item
  GitHub has acquired the following companies:

  \begin{itemize}
  \itemsep1pt\parskip0pt\parsep0pt
  \item
    Easel - In Browser Web Design Tool Built for Collaboration
  \item
    Ordered List - Gauges, Speaker Deck, Harmony
  \end{itemize}
\end{itemize}

\begin{center}\rule{3in}{0.4pt}\end{center}

\begin{itemize}
\itemsep1pt\parskip0pt\parsep0pt
\item
  GitHub hasn't invested in any other companies
\item
  Interestingly enough the first investment in GitHub was for
  \href{http://go.bloomberg.com/tech-deals/2012-07-09-github-takes-100m-in-largest-investment-by-andreessen-horowitz/}{\$100M}
\item
  GitHub has 267 employees, some of which work at their HQ: 88 Colin P
  Kelly Jr St (at Brannan St), San Francisco, CA 94107, United States
\end{itemize}

\begin{center}\rule{3in}{0.4pt}\end{center}

\begin{itemize}
\itemsep1pt\parskip0pt\parsep0pt
\item
  GitHub doesn't have any other locations
\item
  \url{http://github.com}
\item
  \url{http://en.wikipedia.org/wiki/GitHub}
\item
  GitHub does not have any annual reports
\end{itemize}

\section{Communications}\label{communications}

\subsection{Social media for GitHub}\label{social-media-for-github}

\begin{itemize}
\itemsep1pt\parskip0pt\parsep0pt
\item
  \href{https://twitter.com/github}{Twitter}
\item
  \href{https://twitter.com/githubstatus}{Twitter} (Downtime reporting)
\item
  \href{https://www.linkedin.com/company/github}{LinkedIn}
\end{itemize}

\subsection{Communications channels for
GitHub}\label{communications-channels-for-github}

\begin{itemize}
\itemsep1pt\parskip0pt\parsep0pt
\item
  GitHub's press page lists all of their awards that they have received.
  (\url{https://github.com/about/press})
\item
  GitHub also has a blog that has lots of interesting posts.
  (\url{https://github.com/blog})
\item
  They also include a Glossary so writes can better report about what
  the company does.
  (\url{https://help.github.com/articles/github-glossary/})
\end{itemize}

\subsection{GitHub Conference
Participation}\label{github-conference-participation}

\begin{itemize}
\itemsep1pt\parskip0pt\parsep0pt
\item
  GitHub hosts ``GitHub Universe'' which is their own conference about
  working on open source projects as well as gathering tons of people
  working in open source. (\url{http://githubuniverse.com/})
\item
  GitHub also hosts ``CodeConf'' a two day conference about open source,
  best practices, documentation, and collaboration.
  (\url{http://codeconf.com/})
\item
  GitHub also attends several other conferences including GDC.
  (\url{https://github.com/blog/1961-see-you-at-gdc})
\end{itemize}

\section{Community Architecture}\label{community-architecture}

\subsection{Github and Government}\label{github-and-government}

\begin{itemize}
\itemsep1pt\parskip0pt\parsep0pt
\item
  Website: \url{http://government.github.com}
\item
  Repository: \url{http://github.com/github/government.github.com}
\item
  The project is entirely documentation
\end{itemize}

\begin{center}\rule{3in}{0.4pt}\end{center}

\begin{itemize}
\itemsep1pt\parskip0pt\parsep0pt
\item
  \texttt{Github and Government} is a website dedicated to sharing open
  government efforts that involve Github
\item
  Started January 20, 2013, most recent commit March 5, 2015
\item
  Two people (\href{https://github.com/benbalter}{benbalter} and
  \href{https://github.com/jlord}{jlord}) accept all of the
  \href{https://github.com/github/government.github.com/pulls?q=is\%3Apr+is\%3Aclosed}{pull
  requests}
\item
  \href{https://github.com/benbalter}{benbalter} and
  \href{https://github.com/jlord}{jlord} seem to be the only two
  maintainers (BDFLs)
\item
  The original developers have remained with the project the entire
  time.
\end{itemize}

\begin{center}\rule{3in}{0.4pt}\end{center}

\begin{itemize}
\itemsep1pt\parskip0pt\parsep0pt
\item
  Development has been active throughout its lifetime
\item
  The project would continue fine if one of the two main devs were eaten
  by a velociraptor

  \begin{itemize}
  \itemsep1pt\parskip0pt\parsep0pt
  \item
    Not if the
    \href{https://github.com/github/government.github.com/graphs/contributors}{top
    20\% of the contributors} were in a freak bus accident, though -
    that includes both BDFLs
  \end{itemize}
\item
  No onboarding, since most of it is adding information instead of code
\end{itemize}

\begin{center}\rule{3in}{0.4pt}\end{center}

\begin{itemize}
\itemsep1pt\parskip0pt\parsep0pt
\item
  Ruled by a pair of individuals who try to make sure data is complete
  and accurate
\item
  Is this the kind of structure you would enjoy working in? Why, or why
  not?
\item
  This structure is well-suited for this project, but maybe not for a
  code-based project
\end{itemize}

\section{Technology/Product}\label{technologyproduct}

Section adapted from \url{EFF}
\href{http://www.teachingcopyright.org/handout/technology-history-worksheet}{Worksheet}

\subsection{The Technology Behind
GitHub}\label{the-technology-behind-github}

GitHub is powered completely by
\href{http://en.wikipedia.org/wiki/Git_\%28software\%29}{Git}, a
distributed source control system.

\begin{itemize}
\itemsep1pt\parskip0pt\parsep0pt
\item
  Git was originally started by Linus Torvalds, creator of the Linux
  kernel.
\item
  Git was initially started to track the Linux kernel and was then used
  by other projects.
\item
  Lots of people interested in working on the same set of files or
  tracking changes in files could potentially use
  Git.\href{http://en.wikipedia.org/wiki/Git_\%28software\%29}{}
\end{itemize}

\subsection{GitHub Using Git}\label{github-using-git}

\begin{itemize}
\itemsep1pt\parskip0pt\parsep0pt
\item
  Created to be a highly distributed SCM, GitHub based their entire
  product on using Git.
\item
  GitHub now uses git as the tool it uses to track most of their repos
  and interface with GitHub.
\item
  Contrary to popular belief, Git existed before GitHub. GitHub created
  a nice visual front end and a great central area for code repos.
\end{itemize}

\subsection{Copyright Law and Creation of
Git}\label{copyright-law-and-creation-of-git}

\begin{itemize}
\itemsep1pt\parskip0pt\parsep0pt
\item
  The Linux kernel was originally tracked in BitKeeper, a properitary
  SCM.
\item
  Devs of BitKeeper resciended license to use it and the kernel was left
  in the cold.
\item
  Linus became fed up and created his own that would be FOSS so that
  incident could never happen again.
\end{itemize}

\subsection{Controversy over GitHub and
Copyright}\label{controversy-over-github-and-copyright}

\begin{itemize}
\itemsep1pt\parskip0pt\parsep0pt
\item
  GitHub is not open source, leading to an ironic situation that one of
  the most central sites for FOSS software isn't FOSS itself.
\item
  GitHub also has a default license that has been controversial over
  some of its terms.
\end{itemize}
