\subsection{Rationale}\label{rationale}

There are \textbf{so} many subjects you could choose to profile, so why
did you choose this one? What drew you into wanting to know more about
the organization? How did you/your group decide on and agree?

\section{Organizational Details}\label{organizational-details}

\begin{center}\rule{3in}{0.4pt}\end{center}

\begin{itemize}
\itemsep1pt\parskip0pt\parsep0pt
\item
  NGO and Non Profit Organization
\item
  Founded May 15th, 2008
\end{itemize}

\begin{center}\rule{3in}{0.4pt}\end{center}

\begin{itemize}
\itemsep1pt\parskip0pt\parsep0pt
\item
  Marco Presenti Gritti
\item
  Bert Freudenberg
\item
  Simon Schampijer
\item
  Bernardo Innocenti
\item
  Aaron Kaplan
\item
  Christoph Derndorfer
\item
  Tomeu Vizoso
\end{itemize}

\begin{center}\rule{3in}{0.4pt}\end{center}

\begin{itemize}
\itemsep1pt\parskip0pt\parsep0pt
\item
  Not on Stock Market
\item
  0 Acquisitions
\item
  0 Investments
\end{itemize}

\begin{center}\rule{3in}{0.4pt}\end{center}

\begin{itemize}
\itemsep1pt\parskip0pt\parsep0pt
\item
  Walter Bender (President)
\item
  100+ Volunteers
\item
  Boston, MA
\end{itemize}

\begin{center}\rule{3in}{0.4pt}\end{center}

\begin{itemize}
\itemsep1pt\parskip0pt\parsep0pt
\item
  \href{http://sugarlabs.org}{SugarLabs.org}
\item
  \href{http://en.wikipedia.org/wiki/Sugar_Labs}{Wikipedia: SugarLabs}
\end{itemize}

\section{Communications}\label{communications}

\subsection{Social media for Sugar
Labs}\label{social-media-for-sugar-labs}

\begin{itemize}
\itemsep1pt\parskip0pt\parsep0pt
\item
  {[}IRC{]} (Freenode, \#sugar)
\item
  {[}Twitter: @sugarlabs{]}(https://twitter.com/sugarlabs) - 287
  Followers.
\end{itemize}

\subsection{Communications channels for Sugar
Labs}\label{communications-channels-for-sugar-labs}

\begin{itemize}
\itemsep1pt\parskip0pt\parsep0pt
\item
  Sugar has a press page that shows wherever the company or the product
  is mentioned in the news.

  \begin{itemize}
  \itemsep1pt\parskip0pt\parsep0pt
  \item
    \url{http://sugarlabs.org/index.php?template=press}
  \end{itemize}
\end{itemize}

\subsection{Sugar Labs Conference
Participation}\label{sugar-labs-conference-participation}

\begin{itemize}
\itemsep1pt\parskip0pt\parsep0pt
\item
  Sugar does not host any conferences but they do maintain a page where
  they show a bunch of talks by various people in the Sugar community.

  \begin{itemize}
  \itemsep1pt\parskip0pt\parsep0pt
  \item
    \url{http://wiki.sugarlabs.org/go/Marketing_Team/Presentations}
  \end{itemize}
\end{itemize}

\section{Community Architecture}\label{community-architecture}

Your subject likely runs or contributes to one or more Open Source
products or projects. Choose one (or more) of these and answer the
following questions:

\begin{center}\rule{3in}{0.4pt}\end{center}

\begin{itemize}
\itemsep1pt\parskip0pt\parsep0pt
\item
  If applicable, list and provide links to:

  \begin{itemize}
  \itemsep1pt\parskip0pt\parsep0pt
  \item
    irc://irc.freenode.net\#sugar or irc://irc.freenode.net\#fedora-olpc
  \item
    \href{https://github.com/sugarlabs/sugar}{Source Code repository}
  \item
    \href{http://lists.sugarlabs.org/listinfo/sugar-devel}{Mail list -
    Developers}
  \item
    \href{http://wiki.sugarlabs.org/go/Welcome_to_the_Sugar_Labs_wiki}{Documentation}
  \item
    \href{https://www.sugarlabs.org/}{Sugar Labs website}
  \item
    \href{http://planet.sugarlabs.org/}{Blog}
  \end{itemize}
\end{itemize}

\begin{center}\rule{3in}{0.4pt}\end{center}

\begin{itemize}
\itemsep1pt\parskip0pt\parsep0pt
\item
  Sugar is the core component of a worldwide effort to provide every
  child with equal opportunity for a quality education.
\item
  The first commit was Apr 16, 2006. The latest commit was Apr 28, 2015.
\end{itemize}

\begin{center}\rule{3in}{0.4pt}\end{center}

\begin{itemize}
\itemsep1pt\parskip0pt\parsep0pt
\item
  Each core module and activity has it's own project lead and
  co-maintainers.
\item
  There are 62 people who have had commits accepted into the main
  project.
\item
  The core commiters have changed over time. At the beggining there were
  3 major commiters who haven't added anything in awhile.
\item
  Daniel Varvaez is the Lead maintainer for the core sugar project.
\item
  There are different people working on the frontend and the backend.
\end{itemize}

\begin{center}\rule{3in}{0.4pt}\end{center}

\begin{itemize}
\itemsep1pt\parskip0pt\parsep0pt
\item
  Participation seems to be trending down recently but there was a big
  push between October 2014 and January 2015.
\item
  The project would pass the raptor test as there are multiple people
  running each team.
\item
  I don't think this project would pass the bus test. The documentations
  isn't as detailed as it could be.
\end{itemize}

\begin{center}\rule{3in}{0.4pt}\end{center}

\begin{itemize}
\itemsep1pt\parskip0pt\parsep0pt
\item
  Contributing to both the main process and adding new activities has a
  well documented getting started guide.
\item
  The documentation is well broken up so you only need to look at the
  things for the particular thing you're doing, such as the main project
  or an activiy.
\item
  I would contact the maintainers for whichever porition of the project
  I was working on, either by email or hopping into IRC.
\item
  Each individual portion of Sugar has it's own ruling group.
\end{itemize}

\section{Technology/Product}\label{technologyproduct}

\begin{itemize}
\itemsep1pt\parskip0pt\parsep0pt
\item
  Sugar Labs helped produce the operating system and environment that
  runs on the One Laptop Per Child laptops.
\item
  Sugar OS is based off of Fedora and includes the ``Sugar software.''
\item
  Sugar software is typically written in Python which is then turned
  into a Sugar activity that can be installed on OLPC laptops.
\item
  Also provide ``Sugar-on-a-stick'' which is a live USB based off Fedora
  that can be booted on any computer to use the Sugar environment.
\end{itemize}
